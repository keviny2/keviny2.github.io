\documentclass[10pt,a4paper]{article}
\usepackage{xcolor}
\definecolor{ocre}{RGB}{243,102,25}
\usepackage[utf8]{inputenc}
\usepackage{amsmath}
\usepackage{amsfonts}
\usepackage{amssymb}
\usepackage{algpseudocode}
\usepackage{algorithm}

\title{Metropolis Hastings}


\begin{document}
\maketitle 

The Metropolis-Hastings (MH) algorithm is a sampling algorithm that generates samples from a \textit{target distribution} $\pi(\cdot)$. MH is most effective in the case where $\pi(\cdot)$ is intractable. Let's express $\pi(\cdot)$ as the following fraction,
\begin{align*}
\pi(\cdot) = \frac{f(\cdot)}{K}
\end{align*}
where $f(\cdot)$ is some tractable density function and $K$ is an intractable normalizing constant. MH enables us to sample from $\pi(\cdot)$ without knowing anything about the intractable normalizing constant $K$! 
\\
\\
The idea behind MH is to construct a Markov Chain whose stationary distribution is $\pi(\cdot)$. How do we do this? From Markov Chain theory we know that if a Markov Chain is reversible, the detailed balance equation holds,
\begin{align*}
\pi(x)p(x,y) = \pi(y)p(y,x)
\end{align*} 
where $\pi(\cdot)$ is the stationary distribution and $p(x,y)$ is the transition kernel. Hence, if we can find a transition kernel $p(x,y)$ which satisfies this reversibility condition, we can sample from the target distribution $\pi(\cdot)$.
\\
\\
We start by defining some \textbf{tractable} \textit{proposal distribution} $q(x,y)$, which generates a proposal $y$ given some current state $x$ (this is something you pick). Since $q(x,y)$ is a transition kernel, does it satisfy the reversibility condition? Let's derive a general solution. Assume $q(x,y)$ \textbf{does not} satisfy the reversibility condition (if it does, then we just have the Metropolis algorithm). We have the following,
\begin{align*}
\pi(x)q(x,y) > \pi(y)q(y,x) \hspace{1cm} (w.l.g)
\end{align*}
An intuitive interpretation of what this equation is saying is basically we are moving from state $x$ to state $y$ more than we are vice versa; in short, detailed balance is not satisfied. To correct this, we introduce a probability $\alpha(x,y)$ that the move is actually made,
\begin{align*}
\pi(x)q(x,y){\color{ocre}\alpha(x,y)} = \pi(y)q(y,x)
\end{align*}
Rearranging we obtain,
\begin{align*}
\alpha(x,y) = \frac{\pi(y)q(y,x)}{\pi(x)q(x,y)}
\end{align*}
Reiterating, we interpret $\alpha(x,y)$ as the probability of moving from state $x$ to state $y$. Since it is a probability, it cannot exceed $1$ so the final definition is,
\begin{align*}
\alpha(x,y) = \min [\frac{\pi(y)q(y,x)}{\pi(x)q(x,y)}, 1]
\end{align*}
Note that $\alpha(x,y)$ {\color{ocre}\textbf{is}} the transition kernel $p(x,y)$ that we were after! It satisfies the detailed balance equation by definition; however, does it deal with the pesky intractable normalizing constant $K$? Yes! We have that,
		\begin{align*}
		\frac{\pi(y)q(y,x)}{\pi(x)q(x,y)} &= \frac{\frac{f(y)q(y,x)}{K}}{\frac{f(x)q(x,y)}{K}}\\
		&= \frac{f(y)q(y,x)}{f(x)q(x,y)}
		\end{align*}
So we no longer have to worry about $K$ (yay!). Ergo, we can simply use $\alpha(x,y)$ as our transition kernel for our Markov Chain and rest assured that as long as we obtain enough samples, they will be distributed according to the stationary distribution; voila, we have samples from the intractable distribution $\pi(\cdot)$!

\vspace{0.5cm}

\begin{algorithm}
\caption{Metropolis Hastings}
\begin{algorithmic}
\State Initialize $x_0$
\For{$j=1,2,\ldots,N$}
	\State Generate $y$ from $q(x^{(j)},\cdot)$ and $u$ from $U(0,1)$
	\If{$u \leq \alpha(x^{(j)}, y)$}
		\State $x^{(j+1)}=y$
	\Else
		\State $x^{(j+1)}=x^{(j)}$
	\EndIf
\EndFor
\State Return ${x^{(1)},x^{(2)},\ldots,x^{(N)}}$
\end{algorithmic}
\end{algorithm}

\pagebreak

\section{Notes}
\begin{itemize}
	\item $\pi(x)$ is the \textit{target distribution}
	\begin{itemize}
		\item this is the distribution we want to generate samples from
		\item $\pi(x) = \frac{f(x)}{K}$; the normalizing constant $K$ is usually intractable!
	\end{itemize}
	\item $q(x,y)$ is the \textit{proposal distribution} 
	\begin{itemize}
		\item generates a proposal $y$ given the current state $x$
		\item \textbf{you} pick the distribution! (which means you are able to sample from it easily, provided you don't use some intractable distribution)
	\end{itemize}
	\item $\alpha(x,y) = \min [\frac{\pi(y)q(y,x)}{\pi(x)q(x,y)}, 1 ]$ 
	\begin{itemize}
		\item this \textbf{is} the transition kernel that will satisfy detailed balance
		\item probability of moving from state $x$ to state $y$
		\item notice that
		\begin{align*}
		\frac{\pi(y)q(y,x)}{\pi(x)q(x,y)} &= \frac{\frac{f(y)q(y,x)}{K}}{\frac{f(x)q(x,y)}{K}}\\
		&= \frac{f(y)q(y,x)}{f(x)q(x,y)}
		\end{align*}
		so we don't have to worry about the intractable normalizing constant (yay!)		 
	\end{itemize}
	\item $\pi(x)p(x,y) = \pi(y)p(y,x)$ is the \textit{reversibility condition}
	\begin{itemize}
		\item necessarily, $\pi(\cdot)$ is the stationary distribution
		\item $p(x,y)$ is a transition kernel 
	\end{itemize}
\end{itemize}
\end{document}